\PassOptionsToPackage{unicode=true}{hyperref} % options for packages loaded elsewhere
\PassOptionsToPackage{hyphens}{url}
%
\documentclass[]{article}
\usepackage{lmodern}
\usepackage[margin=1in]{geometry}
\usepackage{amssymb,amsmath}
\usepackage{ifxetex,ifluatex}
\ifnum 0\ifxetex 1\fi\ifluatex 1\fi=0 % if pdftex
  \usepackage[T1]{fontenc}
  \usepackage[utf8]{inputenc}
  \usepackage{textcomp} % provides euro and other symbols
\else % if luatex or xelatex
  \usepackage{unicode-math}
  \defaultfontfeatures{Scale=MatchLowercase}
  \defaultfontfeatures[\rmfamily]{Ligatures=TeX,Scale=1}
\fi
% use upquote if available, for straight quotes in verbatim environments
\IfFileExists{upquote.sty}{\usepackage{upquote}}{}
\IfFileExists{microtype.sty}{% use microtype if available
  \usepackage[]{microtype}
  \UseMicrotypeSet[protrusion]{basicmath} % disable protrusion for tt fonts
}{}
\makeatletter
\@ifundefined{KOMAClassName}{% if non-KOMA class
  \IfFileExists{parskip.sty}{%
    \usepackage{parskip}
  }{% else
    \setlength{\parindent}{0pt}
    \setlength{\parskip}{6pt plus 2pt minus 1pt}}
}{% if KOMA class
  \KOMAoptions{parskip=half}}
\makeatother
\usepackage{xcolor}
\IfFileExists{xurl.sty}{\usepackage{xurl}}{} % add URL line breaks if available
\IfFileExists{bookmark.sty}{\usepackage{bookmark}}{\usepackage{hyperref}}
\hypersetup{
  pdfborder={0 0 0},
  breaklinks=true}
\urlstyle{same}  % don't use monospace font for urls
\setlength{\emergencystretch}{3em}  % prevent overfull lines
\providecommand{\tightlist}{%
  \setlength{\itemsep}{0pt}\setlength{\parskip}{0pt}}
\setcounter{secnumdepth}{-2}
% Redefines (sub)paragraphs to behave more like sections
\ifx\paragraph\undefined\else
  \let\oldparagraph\paragraph
  \renewcommand{\paragraph}[1]{\oldparagraph{#1}\mbox{}}
\fi
\ifx\subparagraph\undefined\else
  \let\oldsubparagraph\subparagraph
  \renewcommand{\subparagraph}[1]{\oldsubparagraph{#1}\mbox{}}
\fi

% set default figure placement to htbp
\makeatletter
\def\fps@figure{htbp}
\makeatother


\date{}

\begin{document}

% \hypertarget{header-n388}{%
% \section{Indicadores para Aprendizado de Máquina e seu Uso na Seleção de
% Algoritmos para Análise de Dados}\label{header-n388}}

\title{Indicadores para Aprendizado de Máquina e seu Uso na Seleção de Algoritmos para Análise de Dados}

\author{Alessandro F. Martins\\
Diretor de Consultoria\\
AFM Estratégia de TI para Negócios}

\date{21 de setembro de 2020}
\maketitle

% \emph{Alessandro F. Martins}\\
% \emph{Diretor de Consultoria}\\
% \emph{AFM Estratégia de TI para Negócios}

\hypertarget{header-n392}{%
\renewcommand{\abstractname}{Resumo}
\begin{abstract}
\label{header-n392}}

Cada demanda de negócio pede uma abordagem particular em relação ao
tratamento dos dados para análise preditiva. Conhecendo-se a necessidade
específica, é possível identificar a métrica mais adequada e usá-la como
base para selecionar o algoritmo de aprendizado de máquina que a atenda
da melhor forma.

Uma vez identificada a métrica apropriada, usam-se \emph{metaalgoritmos}
(ou métodos para otimização de \emph{hiperparâmetros}) para buscar, no
espaço de possíveis entradas (ou \emph{espaço de fase}) de cada
algoritmo, aqueles parâmetros que maximizam a capacidade do algoritmo de
produzir os resultados desejados de previsão.

Neste capítulo, nossa tarefa é descrever os conceitos de cada métrica,
em que contextos de negócio elas se fazem mais relevantes, e explorar os
três principais métodos de otimização de hiperparâmetros utilizados por
aplicações no mercado: \emph{Grid Search}, \emph{Randomized Search} e
\emph{Bayes Search}.
\end{abstract}

\hypertarget{header-n397}{%
\subsection{Introdução}\label{header-n397}}

É importante ter-se em mente que \emph{nenhuma métrica é perfeita}. Toda
forma e conceito de medição acaba privilegiando alguma parte da
informação que se quer trabalhar em detrimento de outros aspectos.

Analisemos três situações distintas em suas demandas e peculiaridades.

\hypertarget{header-n401}{%
\paragraph{\texorpdfstring{Análise de desistências
(\emph{Churn})}{Análise de desistências (Churn)}}\label{header-n401}}

No primeiro caso, temos uma empresa de serviços de telecomunicações
(internet, telefonia fixa e móvel, TV por assinatura) que deseja reduzir
sua perda de receita por \emph{Churn} (desistência de clientes, com
possível migração para a concorrência). Para isso, ela deseja obter as
seguintes informações:

\begin{itemize}
\item
  Quais são as razões sob a responsabilidade da empresa que
  preponderantemente estão levando os clientes a deixar a empresa? São
  problemas técnicos, comerciais ou operacionais?
\item
  Existem características específicas dos clientes com tendência a
  desistência? São fatores demográficos (idade, região, possui família,
  renda média), financeiros (pagamentos atrasados ou em atraso) ou
  outros (avaliação de conteúdo constantemente baixa, constante
  utilização do Serviço ao Cliente)?
\item
  Quais são os clientes em minha base mais propensos à desistência no
  próximo ciclo de avaliação de \emph{Churn}? Quais ações devem ser
  tomadas para mantê-los?
\item
  Quais as melhores campanhas a se organizar para garantir a fidelização
  de meus clientes?
\end{itemize}

Consideremos os custos de retenção ativa -- campanhas orientadas
especificamente ao cliente em vias de desistência. Queremos minimizar
esses custos focando apenas neste grupo de clientes, ao mesmo tempo que
levamos em conta o efeito colateral bom de que, fornecendo ao cliente
satisfeito apenas a propaganda que lhe faça sentido, sua satisfação
tende a ser maior. Logo, ao utilizar um determinado algoritmo de
aprendizado de máquina, queremos:

\begin{enumerate}
\def\labelenumi{\arabic{enumi}.}
\item
  selecionar um algoritmo que maximize o indicador que representa a
  \emph{mínima taxa de falsos negativos}
\item
  ajustar corretamente seus hiperparâmetros para o resultado desejado
\end{enumerate}

Logo, precisamos entender qual a importância de que o algoritmo e os
hiperparâmetros selecionados privilegiem uma métrica como a
\emph{precisão}, pois ela indica a razão entre os positivos verdadeiros

Tal

\hypertarget{header-n420}{%
\paragraph{Diagnóstico médico assistido}\label{header-n420}}

No segundo,

\hypertarget{header-n423}{%
\paragraph{Análise de fraudes em cartão de crédito}\label{header-n423}}

Finamente, análise de fraudes em cartão de crédito

\hypertarget{header-n426}{%
\subsection{Métricas para Análise de Dados}\label{header-n426}}

\hypertarget{header-n427}{%
\subsubsection{Nomenclatura inicial:}\label{header-n427}}

\hypertarget{header-n429}{%
\subsubsection{Descrição das métricas}\label{header-n429}}

\begin{itemize}
\item
  \textbf{\emph{Acurácia}}: probabilidade de sucesso do modelo de
  identificar os valores reais. Medida geral de êxito do modelo. Pode
  levar a conclusões errôneas em caso de desbalanceamento entre as
  classes observadas. Corresponde ao cálculo:
\end{itemize}

\[ACC = \frac{TP + TN}{TP + TN + FP + FN}\]

\begin{itemize}
\item
  \textbf{\emph{Precisão}}: probabilidade de sucesso do modelo de fazer
  uma classificação positiva correta. Relevante quando o custo de falsos
  positivos é alto.
\end{itemize}

\[PRC = \frac{TP}{TP + FP}\]

\begin{itemize}
\item
  \textbf{\emph{Sensibilidade}} (ou \emph{Recall}): medida da capacidade
  do modelo de identificar resultados positivos reais. Relevante quando
  o custo de falsos negativos é alto.
\end{itemize}

\[REC = \frac{TP}{TP + FN}\]

\begin{itemize}
\item
  \textbf{\emph{F1}}: média harmônica entre Precisão e Sensibilidade.
  Indica se há um baixo número de valores falsos (positivos e
  negativos), reduzindo a chance de alarmes falsos enquanto identifica
  bem os resultados reais. Mais relevante que a Acurácia, em especial no
  caso de uma distribuição de classes desbalanceada.
\end{itemize}

\[F1 =  2×\frac{PRC × REC}{PRC + REC}   = \frac{2TP}{2TP + FP + FN}\]

\begin{itemize}
\item
  \textbf{\emph{ROC AuC}} (\emph{Área sob a curva característica de
  operação do receptor}): medida da capacidade do modelo de separar
  exemplos positivos e negativos. Resume o \emph{trade-off} entre a taxa
  de positivos verdadeiros e a taxa de falsos positivos.
\item
  \textbf{\emph{PRC AuC}} (\emph{Área sob a curva de precisão/recall}):
  resume o \emph{trade-off} entre a Precisão e a Sensibilidade do
  modelo. Especialmente relevante quando a razão de valores positivos e
  negativos no conjunto de dados é muito desbalanceada.
\item
  \textbf{\emph{MCC}} (\emph{Coeficiente de correlação de Matthews}, ou
  \emph{coeficiente fi (\(\phi\))}): coeficiente de correlação entre as
  classificações observadas e previstas. Um valor de +100\% representa
  uma previsão perfeita, 0 uma previsão não melhor que uma previsão
  aleatória e -100\% um desacordo total entre previsão e observação. 
\end{itemize}

\[MCC = \frac{TP × TN - FP × FN}{\sqrt{(TP + FP)(TP + FN)(TN + FP)(TN + FN)}}\]

\begin{itemize}
\item
  \textbf{\emph{CSI}} (\emph{Índice crítico de sucesso}, também
  conhecido como \textbf{\emph{TS}}: \emph{threat score}): fração de
  positivos verdadeiros que foram corretamente previstos. Equivalente a
  um cálculo de Acurácia em que não se levam em consideração os valores
  negativos reais.
\end{itemize}

\[CSI = \frac{TP}{TP + FP + FN}\]

\begin{itemize}
\item
  \textbf{\emph{FNR}} (\emph{Taxa de perdas}): equivale à probabilidade
  de se classificar uma condição positivo real como negativa.
\end{itemize}

\[FNR = \frac{FN}{TP + FN}\]

\begin{itemize}
\item
  \textbf{\emph{FPR}} (\emph{Taxa de alarmes falsos}): equivale à
  probabilidade de classificar uma condição negativa real como positiva.
\end{itemize}

\[FPR = \frac{FP}{FP + TN}\]

\begin{itemize}
\item
  \textbf{\emph{FDR}} (\emph{Taxa de descobertas falsas}): probabilidade
  do modelo de realizar uma classificação positiva incorreta.
  Complemento da Precisão.
\end{itemize}

\[FDR = \frac{FP}{TP + FP}\]

\begin{itemize}
\item
  \textbf{\emph{FOR}} (\emph{Taxa de omissões falsas}): probabilidade do
  modelo de realizar uma classificação negativa incorreta. Complemento
  do Valor Preditivo Negativo
\end{itemize}

\[FOR = \frac{TN}{TN + FN}\]

\hypertarget{header-n475}{%
\subsection{Referências}\label{header-n475}}

{[}1{]}: Bernard Marr -\/- \emph{What Is The Difference Between
Artificial Intelligence And Machine Learning?}, em Forbes:
https://www.forbes.com/sites/forbesagencycouncil/2018/08/01/do-you-know-the-difference-between-data-analytics-and-ai-machine-learning/\#3a8918cc5878

{[}2{]}: Vance Reavie -\/- \emph{Do You Know The Difference Between Data
Analytics And AI Machine Learning?}, em Forbes:
https://www.forbes.com/sites/bernardmarr/2016/12/06/what-is-the-difference-between-artificial-intelligence-and-machine-learning/\#7ed51bf72742

{[}3{]}: Stuart Russell, Peter Norvig -\/- \emph{Artificial Intelligence
-- A Modern Approach (3\textsuperscript{rd} Edition)}, ISBN:
978-0-13-604259-4

\end{document}
